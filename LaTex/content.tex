\section{Einleitung}
\section{Aufgabenstellung}
\section{Technologierecherche}
\subsection{Distanzmessung}
Optisch:
\begin{itemize}
\item Laser
Vorteile: 	
Für den Innen- und Aussenbereich geeignet. Hervorragende Genauigkeit auch auf grosse Distanzen.
Nachteile: 
Nebel, Regen und starke Sonneneinstrahlung können das Messergebnis beeinträchtigen. Befindet sich der Messpunkt auf einer stark reflektierenden Fläche (Spiegel, Glas, Metalle), kann die Messung gegebenenfalls nicht durchgeführt werden
\item IR
Vorteile:
Bietet konsistente und verlässliche Messungen, welche wenig empfindlich auf äussere Einflüsse wie Temperatur oder Reflektion des zu messenden Objekts sind. Ist zudem kostengünstig.
Nachteile:
Genauigkeit nimmt ab einer Distanz von ca. 80 cm stark ab.
\item Kamera (openCV)
\end{itemize}
Akkustisch:
\begin{itemize}
\item Ultraschall
Vorteile:
Relativ einfach und kostengünstig. Gute Genauigkeit in leeren Innenräumen.
Nachteile:
Hindernisse, schräge Messflächen, Wind und schwankende Temperaturen können das Messergebnis verfälschen. Schallschluckende Materialien können eine Messung unmöglich machen.
\end{itemize}

\section{Testtitel}

Lorem ipsum dolor sit amet, consetetur sadipscing elitr, sed diam nonumy eirmod tempor invidunt ut labore et dolore magna aliquyam erat, sed diam voluptua. At vero eos et accusam et justo duo dolores et ea rebum. Stet clita kasd gubergren, no sea takimata sanctus est Lorem ipsum dolor sit amet. Lorem ipsum dolor sit amet, consetetur sadipscing elitr, sed diam nonumy eirmod tempor invidunt ut labore et dolore magna aliquyam erat, sed diam voluptua. At vero eos et accusam et justo duo dolores et ea rebum. Stet clita kasd gubergren, no sea takimata sanctus est Lorem ipsum dolor sit amet.